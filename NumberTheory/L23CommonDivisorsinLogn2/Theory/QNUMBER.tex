\documentclass[12pt]{article}
\usepackage{enumitem}
\usepackage{fouriernc,amsmath}
\setlist[enumerate]{label*=\arabic*.}
\title{Codechef Problem : QNUMBER}
\date{}
\author{Brij Raj Kishore}

\begin{document}
    \maketitle
    \section{Type 1 Query}
	Given $N$, you have to answer $Q$ queries, in each query you will be given a number K, you have to find count of common divisors of $N$ and $K$ \\
	$N \leq 10^{12}$ \\
	$K \leq 10^{12}$ \\
	$Q \leq 10^5$ \\
	$12$ $3$ \\
	$5 \\ 8 \\ 6 \\$
	$Q1 : 1 \{1\} \\ Q2 : 3 \{1, 2, 4\} \\ Q3 : 4 \{1, 2, 3, 6\}\\$


	\textbf{Naive Approach :} \\
	\begin{enumerate}
		\item Generate list of divisors of $N$ (can be done in $\mathcal{O}(\sqrt{N})$) 
		\item For each query : For each divisor d of N, if it also divides K, then cnt ++
	\end{enumerate}

	\textbf{Complexity Per Query : } $\mathcal{O}(\sqrt[\leftroot{-3}\uproot{3}3]{a})$ \\


	\textbf{Second Approach : } \\
	\begin{enumerate}
		\item Calculate GCD between $N$ and $K$, let it be $G$
		\item Find number of divisors of G
	\end{enumerate}
	
	\textbf{Complexity Per Query : } $\mathcal{O}(\log(max(N, K)) + \sqrt{G})$ \\


	\textbf{Next Approach : } \\

	Find and store Prime factorization of N \\
	$1800 = \{\{2, 3\}, \{3, 2\}, \{5, 2\}\}$ \\
	There are no more than $\log(N)$ primes factors for $N$ \\
	\begin{enumerate}
		\item Using each prime $p$ in prime factorization of $N$, factorize K \\
		$1800 = \{\{2, 3\}, \{3, 2\}, \{5, 2\}\}$ \\
		$200 = \{\{2, 3\}, \{3, 0\}, \{5, 1\}\}$
		
		\item For each prime find the minimum count, and calculate total divisors \\
		$\{\{2, 3\}, \{3, 0\}, \{5, 1\}\}$ \\
		$(3 + 1) \times (0 + 1) \times (1 + 1)$ \\

	\end{enumerate}

	\textbf{Overall Complexity : }
	$\mathcal{O}((\log{N})^2)$

    \section{Type 2 Query}

    For given $K$, find number of divisors of $N$ which are multiple of $K$.

    $K = P_1^{a_1} \times P_2^{a_2} \times P_3^{a_3} \cdots P_m^{a_m}$ \\
    \textbf{Observation 1 : } Let $d$ be a multiple of $K$, then all primes which exist in prime factorization of K must also exist in $d$ and for each prime their power in $d$ must be atlease as much as in $K$ \\

    $K = 60 = \{\{2, 2\}, \{3, 1\}, \{5, 1\}\} \\ M = 180 = \{\{2, 2\}, \{3, 2\}, \{5, 1\}\}$

    \textbf{Observation 2 : } Let $d$ be a multiple of $K$ and divides N, then 
    \begin{enumerate}
        \item $d$ cannot have any prime which is not present in $N$ \\
        Example : \\
        $d = 2 \times 3 \times 3 \times 5$ \\
        $N = 2 \times 3 \times 3 \times 7$
        \item let prime $P$ is present in $d$ with count $x$, and in $N$ with count $y$ then $x \leq y$, in other words $y$ is upper bound. \\
        Example : \\
        $d = 2 \times 3 \times 3 \times 5$ \\
        $N = 2 \times 3 \times 3 \times 5 \times 5$


    \end{enumerate}

    Let $N = 2 \times 3 \times 3 \times 3 \times 5 \times 5$ \\
    $K = 2 \times 3$

    Choices of d : \\
    2    3    5 \\
    1    3    3

    $d1 = 2 \times 3 \times 5 \times 5$ \\
    $d2 = 2 \times 3 \times 3$ \\
    
\end{document}